\chapter{Einleitung}

Unternehmen haben heutzutage einen großen Anspruch an eine IT-Umgebung, welche möglichst fehler- und unterbrechungsfrei funktioniert. Um den Status der Geräte und Systeme in einem Firmennetzwerk bestmöglich überwachen und somit Fehler frühzeitig erkennen zu können, stehen den Systemadministratoren zahlreiche Werkzeuge zur Verfügung, welche die Überwachung auf Fehlerzustände eigenständig übernehmen und den Verantwortlichen somit eine mühselige und monotone Arbeit ersparen. Für die Überwachung der Systeme werden häufig zwei verschiedene Systeme verwendet: ein klassisches Monitoring-System überwacht im festgelegten Sekunden- oder Minutentakt einzelne Systeme auf ihre Erreichbarkeit und Funktionsfähigkeit. Weiterhin können im Monitoring Zustände wie der Füllungsgrad der Datenspeicher oder die Auslastung von CPU- und Arbeitsspeicherressourcen überwacht werden. Das Monitoring-System sorgt so für eine Überwachung der Betriebszustände der Systeme rund um die Uhr. In vielen Netzwerken wird zusätzlich ein Logserver eingesetzt, welcher die anfallenden Systemprotokolle der Geräte im Netzwerk zentral aufnimmt und durchsuchbar macht. Jede Software besitzt eine Art von Fehlerausgabe, welche je nach Einsatzzweck in das zentrale Systemprotokoll des Betriebssystems oder eine Textdatei geschrieben wird. Im Zeitalter von virtualisierten Umgebungen und Microservices hat sich die Anzahl der eigenständigen Systeme in einem Firmennetzwerk vervielfacht. Häufig werden Dienste in die Cloud ausgelagert, da dort Ressourcen und Kosten besser skaliert werden können. Bei der Fehleranalyse eines Dienstausfalls können die Protokolle vom Logserver bezogen werden. Der Logserver bietet weiterhin den Vorteil, dass Zusammenhänge durch den zentralen Standpunkt besser sichtbar werden. Erzeugt ein System aufgrund des vorhergehenden Ausfalls eines anderen Systems eine Fehlermeldung, ist dieser zeitliche Zusammenhang in der kumulierten Ansicht über den Logserver deutlich nachvollziehbar.

Das Produkt Graylog Open \footnote{\url{https://www.graylog.org/products/open-source}} stellt Administratoren eine quelloffene und kostenlose Lösung zur Verfügung, welche eine Vielzahl von Schnittstellen für die Anbindung an Systeme in einem Netzwerk bietet. Für die Bedienung von Graylog und den Zugriff auf die erfassten Protokolle steht eine HTTP-Webschnittstelle für den Webbrowser zur Verfügung. Weiterhin bietet das Produkt eine REST-API für den programmgesteuerten Zugriff an. Mit dieser Schnittstelle können vielseitige benutzerdefinierte Lösungen entwickelt werden, wie eine Aufbereitung der Daten mittels Diagrammen oder der Anschluss der Software an ein self-service Portal. In Graylog können Regeln zur Aufbereitung der erfassten Daten definiert werden. So können spezifische Informationen wie HTTP-Statuscodes eines Webservers oder spezielle Schlüsselwörter mittels regulären Ausdrücken zum Zeitpunkt der Erfassung gefiltert und mittels Attributen global durchsuchbar gemacht werden.

\section{Ziel der Bachelorarbeit}

Es soll die Entwicklung eines Bots für den Telegram-Messenger in der Programmiersprache Python geplant und durchgeführt werden. Der Bot interagiert mit dem Benutzer über gesprochene Sprache indem der Anwender als Administrator eines Netzwerks die gewünschten Informationen zu Systemen im Netzwerk in einer Sprachnachricht beschreibt und der Bot mit einer per Sprachsynthese erstellten Nachricht antwortet. Die Informationsquelle ist eine Installation des Logservers Graylog Open. Es wird vorausgesetzt, dass die Software bereits für den Betrieb im Firmennetzwerk eingerichtet wurde und sämtliche zu überwachende Systeme angeschlossen wurden. Der Anwender kann beispielsweise Informationen zu Webservern im Firmennetzwerk anfordern: "Liefere mir Informationen zu Fehlern bei Webservern im Zeitraum der letzten 24 Stunden". Der Bot soll nun die Inhalte der Nachricht in eine Anfrage an die in Graylog integrierte Suchmaschine umwandeln und dem Anwender das Ergebnis aufbereitet wiedergeben, z.B.: "Es wurden 30 Ereignisse gefunden". Zu den Aufgaben der zu entwickelnden Software gehört die Koordination der Ereignisse inklusive der Behandlung von Fehlerfällen sowie die Kommunikation mit dem Benutzer.

\section{Aufbau der Bachelorarbeit}

In \autoref{cha:grundlagen} werden die technischen Grundlagen erläutert. Hierzu zählen insbesondere die verwendeten Produkte und eine technische Erläuterung, wie die Systeme in der zu entwickelnden Software zusammenarbeiten. In \autoref{cha:planung} wird der Planungsprozess erläutert. Dabei wird auf Modelle des Softwareengineering zurückgegriffen sowie der Programmablauf mit einem UML-Diagramm erläutert. Schließlich werden Entscheidungen erläutert, welche bei der Planung getroffen wurden. In \autoref{cha:implementierung} wird die Implementierung behandelt. Das Kapitel startet mit der Beschreibung der Einrichtung der Softwareentwicklungsumgebung und der Implementierung erster funktionaler Prototypen. Außerdem werden Konzepte erläutert, welche zum Zeitpunkt der Umsetzung der Theorie in die Praxis erstellt wurden. In \autoref{cha:fazit} erfolgt eine Zusammenfassung und ein Ausblick auf mögliche technische Optimierungen.