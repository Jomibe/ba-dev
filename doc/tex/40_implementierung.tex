\chapter{Implementierung}

\section{Struktur}

Um eine modulare Implementierung der Software zu ermöglichen und darüber hinaus den Quellcode getrennt von nicht für die Öffentlichkeit bestimmten Daten (beispielsweise Anmeldedaten für APIs) bereitstellen zu können, ist es notwendig, die Softwareentwicklungsumgebung entsprechend zu gestalten. Dazu wurde wie folgt vorgegangen:

Um die Installation des Programms zu vereinfachen, befindet sich eine Liste mit den Abhängigkeiten zu Bibliotheken von Drittanbietern in der Datei \lstinline{requirements.txt}. Diese Datei kann verwendet werden, um alle für die Ausführung benötigten Abhängigkeiten über den Paketmanager \lstinline{pip} aus dem Python Package Index \lstinline{PyPI} zu laden. Die Datei beinhaltet ebenfalls Angaben zu benötigten Softwareversionen.

Die Programmkonfiguration wird über die Datei \lstinline{constants.py} vorgenommen. Diese beinhaltet Variablen, mit welchen der Ablauf des Programms gesteuert werden kann (beispielsweise können detaillierte Meldungen zum Programmablauf aktiviert oder bestimmte Telegram-Benutzer für den Zugriff auf den Bot autorisiert werden). Des Weiteren beinhaltet die Datei Variablen, welche zur Laufzeit mit Objektdefinitionen der boto3-Bibliothek überschrieben werden. Dies ermöglicht einen durchgehenden Zugriff auf die AWS-API ohne wiederkehrende Authentifizierungen.

Sämtliche Zugangsdaten zu Telegram, Graylog und AWS werden in der Datei \lstinline{.env} hinterlegt. Die Datei wird nicht in das öffentliche GitHub-Repository hochgeladen.

Um die Lesbarkeit und Erweiterbarkeit zu verbessern, wurden die im Python Enhancement Proposal (PEP) Nr. 8 \footnote{https://peps.python.org/pep-0008/} vorgeschlagenen Formatierungsrichtlinien umgesetzt. Für sämtliche Module und Funktionen sind \lstinline{Docstring}-Dokumentationen nach PEP 257 \footnote{https://peps.python.org/pep-0257/} im Quellcode vorhanden.

\section{Module}

Der Quellcode wurde in mehrere Module aufgeteilt:

\begin{itemize}
\item \lstinline{main}: Dieses Modul enthält die Hauptfunktion des Programms und startet die Vorbereitung und den Übergang in den Regelbetrieb.
\item \lstinline{debugging}: Dieses Modul enthält die für die Fehlersuche notwendigen Funktionen. Siehe \autoref{sec:protokollierung}.
\item \lstinline{preparing}: Dieses Modul enthält die für den Startvorgang notwendigen Funktionen, welche nicht Teil eines anderen Moduls sind. Siehe \autoref{sec:startvorgang}
\item \lstinline{telegram}: Dieses Modul enthält Funktionen für den Zugriff auf die Telegram-API. Die HTTP-Abfragen werden mit dem Modul \lstinline{requests} \footnote{https://pypi.org/project/requests/} durchgeführt. Weiterhin sind Funktionen für die Bearbeitung der Antworten der API enthalten. Die Authentifizierung an der API erfolgt über nur dem Entwickler bekannte Zugriffssschlüssel, welche vom Telegram-Bot \lstinline{@BotFather} ausgestellt werden.
\item \lstinline{aws}: Dieses Modul enthält Funktionen für den Zugriff auf AWS. Für die Kommunikation mit der API wird das SDK für die Programmiersprache Python des AWS-Teams \lstinline{boto3} verwendet. Die Verwendung von \lstinline{boto3} gegenüber einer 'low-level' Implementierung der HTTP-Abfragen mit dem Modul \lstinline{requests} bringt mehrere Vorteile, beispielsweise in der Sitzungsverwaltung. Weiterhin bietet die boto3-Bibliothek die Möglichkeit, für ausgesuchte Dienste einen mehr objektorientierten und von API-Funktionen abstrahierten Zugriff zu verwenden \footnote{https://boto3.amazonaws.com/v1/documentation/api/latest/guide/resources.html}.
\item \lstinline{constants}: Dieses Modul beinhaltet Variablen, mit welchen der Ablauf des Programms gesteuert werden kann (beispielsweise können detaillierte Meldungen zum Programmablauf aktiviert oder bestimmte Telegram-Benutzer für den Zugriff auf den Bot autorisiert werden). Des Weiteren sind Variablen als Platzhalter Teil des Moduls, welche zur Laufzeit mit Objektdefinitionen der boto3-Bibliothek überschrieben werden. Dies ermöglicht einen durchgehenden Zugriff auf die AWS-API ohne wiederkehrende Authentifizierungen.
\item \lstinline{graylog}: Dieses Modul beinhaltet Funktionen für den Zugriff auf die Graylog API. Der Zugriff erfolgt ähnlich wie beim Modul \lstinline{telegram} über das Paket \lstinline{requests}. Aus Sicherheitsgründen werden verschiedene Zugangsschlüssel für die Autorisierung und Authentifizierung verwendet \footnote{https://docs.graylog.org/docs/rest-api}. Ein \lstinline{access token} entspricht den Anmeldedaten an der Graylog Webschnittstelle. Mit diesem können zeitlich befristete \lstinline{session token} generiert werden, welche für den Zugriff verwendet werden können. Die Software erkennt anhand der von der API zurückgegebenen Fehlermeldung automatisch, wenn ein nicht gültiger \lstinline{session token} verwendet wurde und beantragt bei Bedarf einen neuen.
\item \lstinline{message_processing}: Dieses Modul enthält interne Funktionen, welche für die Verarbeitung der erhaltenen Nachrichtentexte vorgesehen sind. 
\item \lstinline{store}: Dieses Modul enthält Funktionen, welche für die Persistierung von Informationen notwendig sind. Die Telegram-API verwendet einen Zahlenwert, welcher für jede Aktualisierung um einen Schritt inkrementiert wird. Ruft der Client Aktualisierungen von der API ab, erhält er den derzeitigen Stand als Wert der \lstinline{update_id} (vgl. \autoref{lst:bsp-telegram-api}, Zeile 7). Bei weiteren Anfragen an die API gibt der Client die \lstinline{update_id} an, um nur Aktualisierungen zu erhalten, die seit der letzten Abfrage eingetroffen sind (und daher einen höheren Wert der \lstinline{update_id} haben). Die \lstinline{update_id} wird nach jeder Abfrage in einer Textdatei in das lokale Dateisystem geschrieben.
\end{itemize}

\section{Protokollierung}
\label{sec:protokollierung}

Für die Protokollierung wurde ein selbstentwickeltes Programmmodul erweitert. Als Grundlage diente das Modul \lstinline{debugging} des GitHub-Projekts \lstinline{Jomibe/wireguard-config-generator} \footnote{https://github.com/Jomibe/wireguard-config-generator/blob/main/src/debugging.py}. Die Funktion \lstinline{console()} des Moduls wird verwendet, um die Ausgabe von Statusmeldungen zentral zu koordinieren. Alle Statusmeldungen werden in einer Textdatei hinterlegt, wessen Pfad in der Datei \lstinline{config.py} konfiguriert wird. Ist der \lstinline{DEBUG}-Modus aktiv, erscheinen alle Meldungen zusätzlich auf der Konsole. Ist die Fehlermeldung mit dem Parameter \lstinline{secret} gekennzeichnet, erfolgt keine Protokollierung in der Textdatei. Werden Meldungen auf der Konsole ausgegeben, werden diese gemäß dem Schweregrad farblich markiert: Fehler werden rot markiert, Warnungen gelb, Erfolgsmeldungen grün und Infomeldungen blau. Parameternamen können farblich abgesetzt werden.

\section{Startvorgang}
\label{sec:startvorgang}

Das Modul \lstinline{main} enthält die Methode \lstinline{main()}, welche den Startpunkt des Programms darstellt. In der Vorbereitungsphase (vgl. \autoref{sec:grundsaetzlicher-aufbau}) wird zuerst die Programmbibliothek \lstinline{colorama} initialisiert. Dies ist notwendig, um die Steuerung der Farbausgabe auf der Konsole an das zur Laufzeit verwendete Betriebssystem anzupassen \footnote{https://github.com/tartley/colorama\#initialisation}. Nach der Initialisierung wird die Funktion \lstinline{prepare()} aus dem Modul \lstinline{preparing} aufgerufen. Diese Funktion stellt einen sogenannten 'hook' dar, welcher sämtliche Prüfungen vor Beginn des Regelbetriebs beinhaltet. Wird eine Prüfung nicht erfolgreich abgeschlossen, gibt \lstinline{prepare()} den Wert \lstinline{False} zurück und der Programmablauf wird nach Ausgabe einer detaillierten Fehlermeldung abgebrochen. Falls kein Fehlerstatus zurückgegeben wird, wird zum Regelbetrieb übergegangen. Dieser besteht aus einer Endlosschleife, in welcher die Funktion \lstinline{check_updates()} des Moduls \lstinline{telegram} aufgerufen wird.

\section{Regelbetrieb}

Im Regelbetrieb wird die Funktion \lstinline{getUpdates} der Telegram-API mittels 'long-polling' aufgerufen. Der Wert für den Ablauf der HTTP-Anfrage kann mittels des Parameters \lstinline{TELEGRAM_LONG_POLL_TIMEOUT} verändert werden, standardmäßig sind 30 Sekunden eingestellt.

Eingehende Nachrichten werden in Echtzeit erkannt und weiterverarbeitet. Falls mehrere Nachrichten vorliegen (dies kommt vor, wenn die Software für längere Zeit nicht mit der Telegram-API verbunden war und in der Zwischenzeit mehrere Nachrichten an den Bot gesendet wurden), wird jeweils nur eine Nachricht verarbeitet, da die \lstinline{update_id} nur um jeweils einen Schritt (statt um die Anzahl der neuen Nachrichten) inkrementiert wird. Zuerst wird anhand der Chat ID (vgl. \autoref{lst:bsp-telegram-api}, Zeile 18) ermittelt, ob der Benutzer durch den Administrator für die Verwendung des Bots freigegeben wurde. Hierzu wird der Wert mit der Liste \lstinline{constants.AUTHORIZED_CHAT_IDS} verglichen. Bei positivem Ergebnis erfolgt die Verarbeitung der Nachrichteninhalte: anhand des Aufbaus des JSON-Objekts wird ermittelt, ob es sich um eine Text- oder Sprachnachricht handelt. Der Text einer Nachricht wird direkt durch die Funktion \lstinline{message_processing.process_text_message} verarbeitet. Handelt es sich um eine Sprachnachricht, wird zuerst die Audiodatei von der Telegram Bot-API bezogen und auf dem lokalen Dateisystem abgelegt. Danach erfolgt der Transkriptionsprozess in einer weiteren Funktion, welche den ermittelten Text in einer Zeichenkette zurückgibt. Schließlich wird der Text ebenfalls durch die Funktion \lstinline{message_processing.process_text_message} verarbeitet.

\subsection{Optimierung der Antwortzeit}

Die Bedienung eines Systems mittels Sprache erfordert eine umfangreichere Benutzerführung als die Bedienung eines textbasierten Systems über einen Bildschirm und eine Tastatur. Beim Aufruf einer Webseite in langsamen Netzwerken bietet der Bildschirm durch die Anzeige des Webbrowsers mit diversen Statuselementen eine dauerhafte Möglichkeit für den Benutzer zu prüfen, ob die gewünschte Anfrage eingegangen ist und verarbeitet wird. Eine nur auf Sprache basierende Bedienung bietet keine äquivalente Möglichkeit, dem Benutzer eine Statusübersicht bis zum Abschluss der Anfrage zur Verfügung zu stellen. Um Missverständnisse vorzubeugen und die Bedienung komfortabel zu gestalten, muss das System eine Rückmeldung innerhalb eines durch den Menschen als nicht zu lang empfundenen Zeitraums zurückgeben. Bei der Entwicklung fiel auf, dass die Dauer zwischen Absenden der Sprachnachricht und Erhalt einer Antwort mit den Ergebnissen bis zu 30 Sekunden betrug. Diese Antwortzeit ist nicht akzeptabel. Eine Analyse des Programmablaufs ergab, dass die Transkription und die Sprachsynthese (der Text-To-Speech Prozess) einen Großteil der benötigten Antwortzeit verursachten. Beide Prozesse wurden optimiert, sodass die Verarbeitungszeit der Transkription von 15 Sekunden auf etwa zwei Sekunden und die der Sprachsynthese von 10 Sekunden auf weniger als eine Sekunde verkürzt werden konnte.

Die Zeit von der Erkennung neuer Nachrichten durch den Bot bis zum Abschluss des Versands der Sprachnachricht mit der Antwort beträgt nun etwa fünf Sekunden.

\subsubsection{Transkription}

Die Transkription durch Amazon Transcribe wurde optimiert, indem statt der boto3-Bibliothek das Amazon Transcribe Streaming SDK \footnote{https://github.com/awslabs/amazon-transcribe-streaming-sdk} verwendet wurde. Hierdurch ergaben sich neue Möglichkeiten für die Übermittlung der Audionachricht und des ermittelten Texts durch den Einsatz von HTTP-Streams und asynchronen Funktionen. Mit der boto3-Bibliothek war eine Verarbeitung in Echtzeit nicht möglich. Die Audiodatei musste zuerst in einen S3-Speicher hochgeladen werden. Eine Möglichkeit für die direkte Zuführung der Datei besteht nicht. Im Anschluss musste ein Auftrag in Amazon Transcribe über das \lstinline{client}-Objekt in \lstinline{constants.aws_transcribe_obj} erstellt und die Datei im S3-Speicher damit verknüpft werden. Danach wurde der Auftrag durch AWS verarbeitet. Die Möglichkeit eines 'callbacks' oder der Verwendung von 'long-polling' (vgl. \autoref{sec:telegram-getting-updates}) bestand nicht, daher musste die AWS-API mittels 'polling' durchgehend erneut kontaktiert werden, bis der Auftrag abgeschlossen war. Der transkribierte Text wurde nach der Bearbeitung des Auftrags durch AWS in einem S3-Speicher als JSON-Objekt in einer Textdatei hinterlegt. Nachdem die Datei von S3 bezogen wurde, konnte der Text durch die Software weiterverarbeitet werden.

Der Vorgang wurde durch die Verwendung des Transcribe SDK optimiert, indem die vom Hersteller bereitgestellte Vorlage für eine asynchrone Verarbeitung \footnote{https://github.com/awslabs/amazon-transcribe-streaming-sdk/blob/v0.6.0/examples/simple\_file.py} umgeschrieben wurde. Im Gegensatz zur oben beschriebenen Verwendung von boto3 werden die Audiodaten hierbei direkt Amazon Transcribe über einen HTTP-Stream zugeführt. Dazu wird die zu übertragende Datei, welche sich nach dem Bezug von der Telegram Bot-API auf dem lokalen Dateisystem befindet, mittels des Python Moduls \lstinline{aiofile} in Blöcke mit einer Größe von 16 Kilobyte aufgeteilt und in mehreren Paketen an die AWS-API übertragen. Bereits während der Übertragung und dem Erhalt der ersten Blöcke durch AWS beginnt die Transkription. Die Software implementiert gleichzeitig einen 'event handler', welcher Daten von der AWS API in Echtzeit empfängt und nach Abschluss eines Satzes (dies ist an dem Parameter \lstinline{is_partial}, zu deutsch \lstinline{ist_unvollständig} erkennbar), den Text der Software zuführt und die weitere Verarbeitung durch Abschluss der Funktion auslöst.

Die Entwickler des Transcribe SDK weisen darauf hin, dass sich die Software bislang in einem sehr frühen Entwicklungsstadium befindet. Zum Zeitpunkt der Entwicklung wurde die Version \lstinline{0.6.0} verwendet. Um die Funktionsweise des Bots nicht durch die Abhängigkeit zum Entwicklungsstand der Transcribe SDK zu gefährden, besteht die Möglichkeit, zwischen der klassischen Transkription mit boto3 und der Echtzeit-Transkription mit dem Konfigurationsparameter \lstinline{ENABLE_FAILSAFE_TRANSCRIPTION} zu wechseln.

\subsubsection{Sprachsynthese}

Die Dauer des Text-To-Speech Vorgangs wurde in ähnlicher Weise optimiert wie die der Transkription. Die dazu notwendigen Funktionen waren bereits in der boto3-Bibliothek enthalten. Statt der Funktion \lstinline{start_speech_synthesis_task} wird die Funktion \lstinline{synthesize_speech} verwendet. Dadurch entfällt die Notwendigkeit, die durch AWS erzeugte Audiodatei aus einem S3-Speicher herunterladen zu müssen. Nachdem der TTS-Auftrag zuvor gestartet wurde, musste dieser ebenfalls mittels \lstinline{polling} überwacht werden. Der umzuwandelnde Text wird der API weiterhin als Zeichenkette übergeben. Mit der Funktion \lstinline{synthesize_speech} ist es nun möglich, die Audiodatei aus einem Stream in eine Datei auf dem lokalen Dateisystem zu schreiben. 

\subsection{Verarbeitung des Nachrichtentexts}

Die weitere Verarbeitung des Nachrichtentexts erfolgt nach dem zuvor definierten Modell (vgl. \autoref{sec:syntax}) aus Produktkategorie, Eigenschaft und Zeitraum. Die Schlüsselwörter für die drei zu erkennenden Werte können über \lstinline{constants.KEYWORDS*} festgelegt werden.
