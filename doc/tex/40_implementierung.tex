\chapter{Implementierung}

\section{Grundlegender Aufbau}

Um eine modulare Implementierung der Software zu ermöglichen und darüber hinaus den Quellcode getrennt von nicht für die Öffentlichkeit bestimmten Daten (beispielsweise Anmeldedaten für APIs) bereitstellen zu können, ist es notwendig, die Softwareentwicklungsumgebung entsprechend zu gestalten. Dazu wurde wie folgt vorgegangen:

Um die Installation des Programms zu vereinfachen, befindet sich eine Liste mit den Abhängigkeiten zu Bibliotheken von Drittanbietern in der Datei \lstinline{requirements.txt}. Diese Datei kann verwendet werden, um alle für die Ausführung benötigten Abhängigkeiten über den Paketmanager \lstinline{pip} aus dem Python Package Index \lstinline{PyPI} zu laden. Die Datei beinhaltet ebenfalls Angaben zu benötigten Softwareversionen.

Die Programmkonfiguration wird über die Datei \lstinline{constants.py} vorgenommen. Diese beinhaltet Variablen, mit welchen der Ablauf des Programms gesteuert werden kann (beispielsweise können detaillierte Meldungen zum Programmablauf aktiviert oder bestimmte Telegram-Benutzer für den Zugriff auf den Bot autorisiert werden). Des Weiteren beinhaltet die Datei Variablen, welche zur Laufzeit mit Objektdefinitionen der boto3-Bibliothek überschrieben werden. Dies ermöglicht einen durchgehenden Zugriff auf die AWS-API ohne wiederkehrende Authentifizierungen.

Sämtliche Zugangsdaten zu Telegram, Graylog und AWS werden in der Datei \lstinline{.env} hinterlegt. Die Datei wird nicht in das öffentliche GitHub-Repository hochgeladen.

Um die Lesbarkeit und Erweiterbarkeit zu verbessern, wurden die im Python Enhancement Proposal (PEP) Nr. 8 \footnote{https://peps.python.org/pep-0008/} vorgeschlagenen Formatierungsrichtlinien umgesetzt. Für sämtliche Module und Funktionen sind \lstinline{Docstring}-Dokumentationen nach PEP 257 \footnote{https://peps.python.org/pep-0257/} im Quellcode vorhanden.

\subsection{Module}

Der Quellcode wurde in mehrere Module aufgeteilt:

\begin{itemize}
\item \lstinline{main}: Dieses Modul enthält die Hauptfunktion des Programms und startet die Vorbereitung und den Übergang in den Regelbetrieb.
\item \lstinline{debugging}: Dieses Modul enthält die für die Fehlersuche notwendigen Funktionen. Siehe \autoref{sec:protokollierung}.
\item \lstinline{preparing}: Dieses Modul enthält die für den Startvorgang notwendigen Funktionen, welche nicht Teil eines anderen Moduls sind. Siehe \autoref{sec:startvorgang}
\item \lstinline{telegram}: Dieses Modul enthält Funktionen für den Zugriff auf die Telegram-API. Die HTTP-Abfragen werden mit dem Modul \lstinline{requests} \footnote{https://pypi.org/project/requests/} durchgeführt. Weiterhin sind Funktionen für die Bearbeitung der Antworten der API enthalten. Die Authentifizierung an der API erfolgt über nur dem Entwickler bekannte Zugriffssschlüssel, welche vom Telegram-Bot \lstinline{@BotFather} ausgestellt werden.
\item \lstinline{aws}: Dieses Modul enthält Funktionen für den Zugriff auf AWS. Für die Kommunikation mit der API wird das SDK für die Programmiersprache Python des AWS-Teams \lstinline{boto3} verwendet. Die Verwendung von \lstinline{boto3} gegenüber einer \lstinline{low-level} Implementierung der HTTP-Abfragen mit dem Modul \lstinline{requests} bringt mehrere Vorteile, beispielsweise in der Sitzungsverwaltung. Weiterhin bietet \lstinline{boto3} die Möglichkeit, für ausgesuchte Dienste einen mehr objektorientierten und von API-Funktionen abstrahierten Zugriff zu verwenden \footnote{https://boto3.amazonaws.com/v1/documentation/api/latest/guide/resources.html}.
\item \lstinline{constants}: Dieses Modul beinhaltet Variablen, mit welchen der Ablauf des Programms gesteuert werden kann (beispielsweise können detaillierte Meldungen zum Programmablauf aktiviert oder bestimmte Telegram-Benutzer für den Zugriff auf den Bot autorisiert werden). Des Weiteren sind Variablen als Platzhalter Teil des Moduls, welche zur Laufzeit mit Objektdefinitionen der boto3-Bibliothek überschrieben werden. Dies ermöglicht einen durchgehenden Zugriff auf die AWS-API ohne wiederkehrende Authentifizierungen.
\item \lstinline{graylog}: Dieses Modul beinhaltet Funktionen für den Zugriff auf die Graylog API. Der Zugriff erfolgt ähnlich wie beim Modul \lstinline{telegram} über das Paket \lstinline{requests}. Aus Sicherheitsgründen werden verschiedene Zugangsschlüssel für die Autorisierung und Authentifizierung verwendet \footnote{https://docs.graylog.org/docs/rest-api}. Ein \lstinline{access token} entspricht den Anmeldedaten an der Graylog Webschnittstelle. Mit diesem können zeitlich befristete \lstinline{session token} generiert werden, welche für den Zugriff verwendet werden können. Die Software erkennt anhand der von der API zurückgegebenen Fehlermeldung automatisch, wenn ein nicht gültiger \lstinline{session token} verwendet wurde und beantragt bei Bedarf einen neuen.
\item \lstinline{message_processing}: Dieses Modul enthält interne Funktionen, welche für die Verarbeitung der erhaltenen Nachrichtentexte vorgesehen sind. 
\item \lstinline{store}: Dieses Modul enthält Funktionen, welche für die Persistierung von Informationen notwendig sind. Die Telegram-API verwendet einen Zahlenwert, welcher für jede Aktualisierung um einen Schritt inkrementiert wird. Ruft der Client Aktualisierungen von der API ab, erhält er den derzeitigen Stand als Wert der \lstinline{update_id} (vgl. \autoref{lst:bsp-telegram-api}, Zeile 7). Bei weiteren Anfragen an die API gibt der Client die \lstinline{update_id} an, um nur Aktualisierungen zu erhalten, die seit der letzten Abfrage eingetroffen sind (und daher einen höheren Wert der \lstinline{update_id} haben). Die \lstinline{update_id} wird nach jeder Abfrage in einer Textdatei in das lokale Dateisystem geschrieben.
\end{itemize}

\subsection{Protokollierung}
\label{sec:protokollierung}

Für die Protokollierung wurde ein selbstentwickeltes Programmmodul erweitert. Als Grundlage diente das Modul \lstinline{debugging} des GitHub-Projekts \lstinline{Jomibe/wireguard-config-generator} \footnote{https://github.com/Jomibe/wireguard-config-generator/blob/main/src/debugging.py}. Die Funktion \lstinline{console()} des Moduls wird verwendet, um die Ausgabe von Statusmeldungen zentral zu koordinieren. Alle Statusmeldungen werden in einer Textdatei hinterlegt, wessen Pfad in der Datei \lstinline{config.py} konfiguriert wird. Ist der \lstinline{DEBUG}-Modus aktiv, erscheinen alle Meldungen zusätzlich auf der Konsole. Ist die Fehlermeldung mit dem Parameter \lstinline{secret} gekennzeichnet, erfolgt keine Protokollierung in der Textdatei. Werden Meldungen auf der Konsole ausgegeben, werden diese gemäß dem Schweregrad farblich markiert: Fehler werden rot markiert, Warnungen gelb, Erfolgsmeldungen grün und Infomeldungen blau. Parameternamen können farblich abgesetzt werden.

\subsection{Startvorgang}
\label{sec:startvorgang}

Das Modul \lstinline{main} enthält die Methode \lstinline{main()}, welche den Startpunkt des Programms darstellt. In der Vorbereitungsphase (vgl. \autoref{sec:grundsaetzlicher-aufbau}) wird zuerst die Programmbibliothek \lstinline{colorama} initialisiert. Dies ist notwendig, um die Steuerung der Farbausgabe auf der Konsole an das zur Laufzeit verwendete Betriebssystem anzupassen \footnote{https://github.com/tartley/colorama\#initialisation}. Nach der Initialisierung wird die Funktion \lstinline{prepare()} aus dem Modul \lstinline{preparing} aufgerufen. Diese Funktion stellt einen sogenannten \lstinline{Hook} dar, welcher sämtliche Prüfungen vor Beginn des Regelbetriebs beinhaltet. Wird eine Prüfung nicht erfolgreich abgeschlossen, gibt \lstinline{prepare()} den Wert \lstinline{False} zurück und der Programmablauf wird nach Ausgabe einer detaillierten Fehlermeldung abgebrochen. Falls kein Fehlerstatus zurückgegeben wird, wird zum Regelbetrieb übergegangen. Dieser besteht aus einer Endlosschleife, in welcher die Funktion \lstinline{check_updates()} des Moduls \lstinline{telegram} aufgerufen wird.

\subsection{Regelbetrieb}

Im Regelbetrieb wird die Funktion \lstinline{getUpdates} der Telegram-API mittels \lstinline{long-polling} aufgerufen. Der Wert für den Ablauf der HTTP-Anfrage kann mittels des Parameters \lstinline{TELEGRAM_LONG_POLL_TIMEOUT} verändert werden, standardmäßig sind 30 Sekunden eingestellt.
