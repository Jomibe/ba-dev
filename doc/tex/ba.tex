\documentclass[a4paper, 11pt, toc=listof, toc=bib, twoside=off]{scrbook}

% Quelle: https://www.fh-swf.de/media/neu_np/fb_in/professorinnen_3/giefers/pub/docs/LaTeX-Vorlage_Abschlussarbeit.zip
% Vorlage fuer Abschlussarbeiten
% Version 2022-05-17
% Getestet mit TeX Live; Kodierung: UTF8

% Hier Daten eintragen:
\newcommand{\name}{Jonas Michel Berger}
\newcommand{\abschlussarbeit}{Bachelorarbeit}
\newcommand{\hochschule}{Fachhochschule Südwestfalen}
\newcommand{\datum}{2. November 2022}
\newcommand{\titeldeutsch}{Entwicklung eines Telegram-Bots zur Abfrage von Informationen aus der Software Graylog per Sprachnachricht}
\newcommand{\titelenglisch}{Development of a Telegram Bot to Retrieve Information From Graylog Software via Voice Message}
\newcommand{\erstpruefer}{Prof. Dr. Hans-Georg Eßer}
\newcommand{\zweitpruefer}{Prof. Dr. Heiner Giefers}
% Ende der Daten

\usepackage[inner=25mm, outer=25mm, top=25mm, bottom=25mm]{geometry}
\usepackage[utf8]{inputenc}
\usepackage[T1]{fontenc}
\usepackage{lmodern}
\usepackage[ngerman]{babel}

% https://tex.stackexchange.com/questions/150945/automatic-german-quotation-marks
\usepackage{xspace}

\let\oldquote'
\newif\ifquoteopen
\catcode`\'=\active
\makeatletter
% we have to redefine \pr@m@s to use an active '
\def\pr@m@s{%
  \ifx'\@let@token
    \expandafter\pr@@@s
  \else
    \ifx^\@let@token
      \expandafter\expandafter\expandafter\pr@@@t
    \else
      \egroup
    \fi
  \fi}
\protected\def'{%
  \ifmmode
    \expandafter\active@math@prime
  \else
    \expandafter\active@text@prime
  \fi}
\def\active@text@prime{%
   \@ifnextchar'{%
     \ifquoteopen
       \global\quoteopenfalse\grqq\expandafter\@gobble
     \else
       \global\quoteopentrue\glqq\expandafter\@gobble
     \fi
   }{%
     \ifquoteopen
       \global\quoteopenfalse\grq\xspace
     \else
       \global\quoteopentrue\glq
     \fi
   }%
}
\makeatother
\usepackage[style=alphabetic, backend=biber, citestyle=ieee, bibstyle=numeric]{biblatex}
\addbibresource{literatur.bib}
\usepackage{csquotes}
\MakeOuterQuote{"}
%\usepackage{scrlayer-scrpage}\lohead{\rightmark}\rehead{\leftmark}\ohead{\pagemark}
\usepackage{booktabs}
\usepackage{microtype}
\usepackage{graphicx}
\usepackage{scrhack}
\usepackage{hyperref}
\usepackage{placeins}
\parindent0pt\parskip6pt

\usepackage{listings}
\renewcommand{\lstlistingname}{Listing}
\renewcommand{\lstlistlistingname}{Listingverzeichnis}
\lstset{basicstyle=\small\ttfamily, breaklines=true, numbers=left}
\lstset{literate=%
    {Ö}{{\"O}}1
    {Ä}{{\"A}}1
    {Ü}{{\"U}}1
    {ß}{{\ss}}1
    {ü}{{\"u}}1
    {ä}{{\"a}}1
    {ö}{{\"o}}1
    {~}{{\textasciitilde}}1
}

\usepackage[normalem]{ulem}

\addto\extrasngerman{
%\renewcommand{\figureautorefname}{foo}
%\renewcommand{\tableautorefname}{bar}
%\renewcommand{\sectionautorefname}{foobar}
\renewcommand{\subsectionautorefname}{Abschnitt}
}

\linespread{1.15}

\graphicspath{ {./img/} }

\begin{document}
\frontmatter
\begin{titlepage}
\begin{center}
\begin{center}
\includegraphics{01_Logo-CMYK}
\end{center}

\vspace*{10mm}
\huge
\textbf{\titeldeutsch}

\vspace{10mm}
\Large
\titelenglisch

\vspace{15mm}
\LARGE
\textsc{\abschlussarbeit}

\vspace{20mm}
\large
\name

\hochschule

\datum
\end{center}
\end{titlepage}

\clearpage

\normalsize\normalfont

\thispagestyle{plain}
\begin{tabular}{ll}
Autor: & \name \\
Referent: & \erstpruefer \\
Korreferent: & \zweitpruefer \\
Eingereicht: & \datum
\end{tabular}

\chapter*{Zusammenfassung}
\addcontentsline{toc}{chapter}{Zusammenfassung}

Das Thema dieser Abschlussarbeit ist die Entwicklung eines virtuellen Assistenten (Bot) für den Telegram Messenger, welcher Netzwerkadministratoren bei ihrer täglichen Arbeit durch die Möglichkeit unterstützen soll, in Echtzeit Informationen zu ihrem verwalteten Netzwerk abzurufen. Dabei wird auf die Software Graylog zurückgegriffen, welche die zentrale Verwaltung von anfallenden Systemprotokollen für ein Netzwerk übernimmt. Der Systemadministrator kann vordefinierte Abfragen in der Bot-Software hinterlegen und verknüpfen, sodass diese auf Anfrage jederzeit an die in Graylog integrierte Suchmaschine übermittelt werden können. Die Kommunikation zwischen Bot und Anwender erfolgt über Sprachnachrichten. Um die Nachrichten verarbeiten zu können, wandelt der Bot eingehende Sprachnachrichten mit einer Spracherkennung in Text um.

\chapter*{Erklärung}

Ich erkläre hiermit, dass ich die vorliegende Arbeit selbstständig verfasst und dabei keine anderen als die angegebenen Hilfsmittel benutzt habe. Sämtliche Stellen der Arbeit, die im Wortlaut oder dem Sinn nach Werken anderer Autoren entnommen sind, habe ich als solche kenntlich gemacht. Die Arbeit wurde bisher weder gesamt noch in Teilen einer anderen Prüfungsbehörde vorgelegt und auch noch nicht veröffentlicht.

\bigskip
\noindent
\datum

\vspace{25mm}

\noindent
\name

\tableofcontents

\mainmatter
%\ofoot*{}
\chapter{Einleitung}

Unternehmen haben heutzutage einen großen Anspruch an eine IT-Umgebung, welche möglichst fehler- und unterbrechungsfrei funktioniert. Um den Status der Geräte und Systeme in einem Firmennetzwerk bestmöglich überwachen und somit Fehler frühzeitig erkennen zu können, stehen den Systemadministratoren zahlreiche Werkzeuge zur Verfügung, welche die Überwachung auf Fehlerzustände eigenständig übernehmen und den Verantwortlichen somit eine mühselige und monotone Arbeit ersparen. Für die Überwachung der Systeme werden häufig zwei verschiedene Systeme verwendet. 

Ein klassisches Monitoring-System überwacht im festgelegten Sekunden- oder Minutentakt einzelne Systeme auf ihre Erreichbarkeit und Funktionsfähigkeit. Weiterhin können im Monitoring Zustände wie der Füllungsgrad der Datenspeicher oder die Auslastung von CPU- und Arbeitsspeicherressourcen überwacht werden. Das Monitoring-System sorgt so für eine Überwachung der Betriebszustände der Systeme rund um die Uhr. 

In vielen Netzwerken wird zusätzlich ein Logserver eingesetzt, welcher die anfallenden Systemprotokolle der Geräte im Netzwerk zentral aufnimmt und durchsuchbar macht. Jede Software besitzt eine Art von Fehlerausgabe, welche je nach Einsatzzweck in das zentrale Systemprotokoll des Betriebssystems oder eine Textdatei geschrieben wird. Im Zeitalter von virtualisierten Umgebungen und Microservices hat sich die Anzahl der eigenständigen Systeme in einem Firmennetzwerk vervielfacht. Häufig werden Dienste in die Cloud ausgelagert, da dort Ressourcen besser skaliert werden können und dabei keine einmaligen und fortlaufenden Kosten für die Anschaffung und Wartung neuer Hardware entstehen. Bei der Fehleranalyse eines Dienstausfalls können die Protokolle vom Logserver bezogen werden. Erzeugt ein System aufgrund des vorhergehenden Ausfalls eines anderen Systems eine Fehlermeldung, ist dieser zeitliche Zusammenhang in der kumulierten Ansicht über den Logserver deutlich nachvollziehbar.

Das Produkt Graylog Open\footnote{\url{https://www.graylog.org/products/open-source}} stellt Administratoren eine quelloffene und kostenlose Lösung zur Verfügung, welche eine Vielzahl von Schnittstellen für die Anbindung an Systeme in einem Netzwerk bietet. Für die Bedienung von Graylog und den Zugriff auf die erfassten Protokolle steht eine HTTP-Webschnittstelle für den Webbrowser zur Verfügung. Weiterhin bietet das Produkt eine REST-API für den programmgesteuerten Zugriff an. Mit dieser Schnittstelle können vielseitige benutzerdefinierte Lösungen entwickelt werden, wie eine Aufbereitung der Daten mittels Diagrammen oder der Anschluss der Software an ein Selfservice Portal. In Graylog können Regeln zur Aufbereitung der erfassten Daten definiert werden. So können spezifische Informationen wie HTTP-Statuscodes eines Webservers oder spezielle Schlüsselwörter mittels regulären Ausdrücken zum Zeitpunkt der Erfassung gefiltert und mittels Attributen global durchsuchbar gemacht werden.

\section{Ziel der Bachelorarbeit}

Es soll die Entwicklung eines Bots für den Telegram-Messenger in der Programmiersprache Python geplant und durchgeführt werden. Der Bot interagiert mit dem Benutzer über gesprochene Sprache indem der Anwender als Administrator eines Netzwerks die gewünschten Informationen zu Systemen im Netzwerk in einer Sprachnachricht beschreibt und der Bot mit einer per Sprachsynthese erstellten Nachricht antwortet. Die Informationsquelle ist eine Installation des Logservers Graylog Open. Es wird vorausgesetzt, dass die Software bereits für den Betrieb im Firmennetzwerk eingerichtet wurde und sämtliche zu überwachende Systeme angeschlossen wurden. Der Anwender kann beispielsweise Informationen zu Webservern im Firmennetzwerk anfordern: "Liefere mir Informationen zu Fehlern bei Webservern im Zeitraum der letzten 24 Stunden". Der Bot soll nun die Inhalte der Nachricht in eine Anfrage an die in Graylog integrierte Suchmaschine umwandeln und dem Anwender das Ergebnis aufbereitet wiedergeben, z.B.: "Es wurden 30 Ereignisse gefunden". Zu den Aufgaben der zu entwickelnden Software gehört die Koordination der Ereignisse inklusive der Behandlung von Fehlerfällen sowie die Kommunikation mit dem Benutzer.

\section{Aufbau der Bachelorarbeit}

In \autoref{cha:grundlagen} werden die technischen Grundlagen erläutert. Hierzu zählen insbesondere die verwendeten Produkte und eine technische Erläuterung, wie die Systeme in der zu entwickelnden Software zusammenarbeiten. In \autoref{cha:planung} wird der Planungsprozess erläutert. Dabei wird auf Modelle des Softwareengineering zurückgegriffen sowie der Programmablauf mit einem UML-Diagramm erläutert. Schließlich werden Entscheidungen erläutert, welche bei der Planung getroffen wurden. In \autoref{cha:implementierung} wird die Implementierung behandelt. Das Kapitel startet mit der Beschreibung der Einrichtung der Softwareentwicklungsumgebung und der Implementierung erster funktionaler Prototypen. Außerdem werden Konzepte erläutert, welche zum Zeitpunkt der Umsetzung der Theorie in die Praxis erstellt wurden. In \autoref{cha:fazit} erfolgt eine Zusammenfassung und ein Ausblick auf mögliche technische Optimierungen.
\chapter{Grundlagen}
\label{cha:grundlagen}

In diesem Kapitel werden Technologien und Produkte erläutert, auf welchen die zu implementierende Software basiert.

\section{REST}

REST ist eine Abkürzung für 'Representational State Transfer' \cite[S. 76 ff.]{rest}. Es handelt sich hierbei um ein Vorbild für einen Softwarearchitekturstil, welches in der Entwicklung verteilter Systeme Anwendung findet. REST-APIs werden für viele Dienste des Internets meist unsichtbar für die Anwender für die Kommunikation zwischen informationstechnischen Systemen verwendet. In der englischen Sprache wird häufig das Adjektiv 'RESTful' für Ressourcen verwendet, welche die REST-Prinzipien anwenden \cite[S. 277]{swdev}.

\subsection{Anwendung in der Praxis}

Heutzutage stellen viele Dienste HTTP-APIs zur Verfügung, welche als 'RESTful' bezeichnet werden können. Die Kommunikation mit den Programmierschnittstellen beschränkt sich dabei auf die HTTP- und HTTPS-Protokolle, welche auch für den Abruf von Webinhalten im Browser verwendet werden. Durch die hohe Verbreitung wird die Implementierung neuer vernetzter Applikationen für Softwareentwickler stark vereinfacht.

Der Zugriff auf HTTP-APIs erfolgt analog zum Zugriff auf Webseiten. Je nach Anforderung werden die HTTP-Methoden 'GET', 'POST' und weitere verwendet \cite[S. 279]{swdev}. Die von der HTTP-API beziehbaren Informationen werden mit der URL (Uniform Resource Locator) lokalisiert und angefordert. Zusätzlich können Informationen über Header (Kopfzeilen) und Cookies ausgetauscht werden, um beispielsweise Zustandsinformationen (Status des Warenkorbs in einem Webshop, Identifikationsnummer des angemeldeten Benutzers) übertragen zu können.

\begin{lstlisting}[caption={Beispiel eines Aufrufs der Graylog REST-API.}, label=bsp-rest-api, numbers=none, xleftmargin=6mm]
GET http://10.10.12.1:9000/api/cluster
\end{lstlisting}
\begin{lstlisting}[xleftmargin=6mm]
{
  "12ac8e44-0c59-4c38-bd5e-4fe247a55893": {
    "facility": "graylog-server",
    "codename": "Noir",
    "node_id": "12ac8e44-0c59-4c38-bd5e-4fe247a55893",
    "cluster_id": "8421ad08-fe62-4a04-8576-a81cab1a9c55",
    "version": "4.3.6+36120cf",
    "started_at": "2022-09-10T23:15:05.435Z",
    "hostname": "graylog.home.arpa",
    "lifecycle": "running",
    "lb_status": "alive",
    "timezone": "Europe/Berlin",
    "operating_system": "Linux 5.4.0-125-generic",
    "is_processing": true
  }
}
\end{lstlisting}

Die Antwort der HTTP-API erfolgt in vielen Fällen maschinenlesbar in der Sprache JSON \cite[S. 281]{swdev}, wie in \autoref{bsp-rest-api} ab der ersten Zeile zu sehen.

\section{Telegram Messenger}

Der Dienst Telegram bietet seinen Nutzern eine Möglichkeit um kostenlos miteinander u.\,a. über Apps für die Mobilbetriebssysteme Android und iOS zu kommunizieren. Telegram hatte nach nach eigenen Angaben weltweit im Juni 2022 erstmals über 700 Millionen monatliche Nutzer und befand sich unter den fünf Apps mit den höchsten Downloadzahlen\footnote{\url{https://telegram.org/blog/700-million-and-premium}}. Der Dienst stellt damit eine beliebte Alternative zu anderen Messengern wie WhatsApp oder Signal dar. Ein Alleinstellungsmerkmal von Telegram gegenüber WhatsApp oder Signal sind die Bots und die dazugehörige API.

\subsection{Bots}

Der Begriff Bot stammt von dem englischen Begriff 'robot' ab und bezeichnet ein Programm, welches einfache und wiederkehrende Aufgaben selbstständig erledigt. Bots werden heutzutage an vielen Stellen eingesetzt, beispielsweise in Telefonwarteschleifen oder in Webshops in Form von virtuellen Beratern für den Verkauf oder den Service. Der Telegram Messenger bietet Softwareentwicklern die Möglichkeit, mittels einer HTTP-API einen Bot auf vielfältige Weise an eigene Software anzubinden. Der Bot agiert Benutzern auf der Telegram-Plattform gegenüber wie ein normaler Verbindungspartner mit dem Unterschied, dass dieser zuerst mit einer Initialnachricht ('/start') vom Benutzer aktiviert werden muss. So wird verhindert, dass  Benutzer ohne Zustimmung kontaktiert werden können. Bots können auch zu Gruppen hinzugefügt werden. Dem Entwickler stehen zahlreiche und ausführlich dokumentierte API-Funktionen\footnote{\url{https://core.telegram.org/bots/api\#available-methods}} zur Verfügung, sodass die vom Bot ausgehende Kommunikation nicht auf Textnachrichten limitiert ist. Bots stehen u.\,a. folgende Möglichkeiten zur Verfügung:

\begin{itemize}
\item Senden und Empfangen von Audio-, Bild- und Videodateien, Sprachnachrichten und Standortdaten
\item Erstellen von Umfragen in einer Gruppe
\item Abfrage von Informationen der aktiven Kommunikationspartner: Profilbild, Name, Zeitpunkt der letzten Erreichbarkeit (sofern vom Benutzer freigegeben)
\item Gruppenadministration: Setzen und Entfernen von MOTD-Nachrichten (dt. "Mitteilungen des Tages")
\item Vereinfachte Bedienung durch vom Entwickler definierte Antwortmöglichkeiten, welche durch den Anwender direkt mittels Buttons gewählt werden können
\end{itemize}

Aufgrund der weitreichenden Möglichkeiten, der kostenlosen Benutzbarkeit und der für Anwendungsfälle mit bis zu 30 API-Zugriffen pro Sekunde ausreichenden Durchsatzratenbegrenzung\footnote{\url{https://core.telegram.org/bots/faq\#my-bot-is-hitting-limits-how-do-i-avoid-this}} steht eine Vielzahl von Bots für die Benutzer der Plattform zur Verfügung. Als Beispiele seien genannt der Bot "WDR aktuell" <\lstinline{@WDRaktuell_bot}> für den Abruf von regionalen Informationen in Nordrhein-Westfalen, der von Google angebotene und daher als "verifiziert"\footnote{\url{https://telegram.org/verify}} markierte "Gmail Bot" <\lstinline{@GmailBot}> für den Nachrichtenabruf und das Senden von Antworten für Gmail sowie der Bot IFTTT ("if this then that") <\lstinline{@IFTTT}>, welcher die Automatisierung und Verknüpfungen von Webanwendungen erleichtert.

\begin{figure}[h!]
\centering
\includegraphics[scale=0.7]{telegram-bot-beispiel}
\caption{Interaktion mit dem Bot von WDR aktuell in Telegram für macOS.}
\end{figure}

\begin{figure}[h!]
\centering
\includegraphics[scale=0.7]{verified_ifttt_bot}
\caption{Verifizierter IFTTT-Bot in der Suche von Telegram für macOS.}
\end{figure}

\subsubsection{Beziehen von Aktualisierungen von der Telegram API}
\label{sec:telegram-getting-updates}

Laut der technischen Dokumentation der Telegram API\footnote{\url{https://core.telegram.org/bots/api\#getting-updates}} stehen zwei Möglichkeiten zur Verfügung, um Aktualisierungen zu erhalten: Long-polling mit der API-Methode \lstinline{getUpdates()} oder die Verwendung eines Webhooks. Die beiden Möglichkeiten verwenden unterschiedliche Konzepte und bieten damit verschiedene Vor- und Nachteile. 

Im Fachbereich der Informatik steht der Begriff Polling für eine dauerhafte wiederkehrende Abfrage von Informationen bei einem Dienst. Diese Technik beansprucht viel CPU-Zeit eines Systems und wird aufgrund des geringen Gegenwerts der meist leer ausgehenden Abfragen als teuer bezeichnet. Long-polling beschreibt eine Technik, bei welcher der Client einer HTTP-Abfrage einen verlängerten Zeitraum bis zum Erhalt der Antwort vom Server akzeptiert, ohne die Abfrage aufgrund einer Zeitüberschreitung abzubrechen. Die Länge des Zeitraums ist variabel. Long-polling bietet damit gegenüber dem klassischen Polling den Vorteil, dass die benötigte Rechenzeit durch die verlängerten Abfragezeiträume stark reduziert wird. 

Die Software- und Betriebssystementwicklung bietet eine Technik, um die Verwendung von teurem Polling zu umgehen: die Verwendung von Interrupts oder Callbacks (dt. Rückruf). Hierbei erhält der Absender der Anfrage eine Rückmeldung, sobald die Antwort vorliegt. Der Vorteil dieser Technik ist, dass nach dem Absenden der Anfrage keine CPU-Zeit seitens des Absenders notwendig ist. Es muss jedoch eine Möglichkeit vorhanden sein, den Absender im Falle einer eingehenden Antwort zum Vorgang zurückzuführen, sodass die weitere Bearbeitung möglich wird. Im Falle der Anbindung der Telegram API übernimmt dies der Webhook. Der Webhook muss (durch die Telegram-API) öffentlich aus dem Internet erreichbar und durch den Verwalter des Bots im Vorhinein durch eine entsprechende URL definiert worden sein. Wird eine Nachricht an den Bot gesendet, prüft die Telegram-API intern, ob Definitionen für Webhooks vorliegen. Im positiven Fall sendet die Telegram API eine HTTP-Anfrage an die URL des Webhooks mit den Details zur eingegangenen Nachricht. Nach Erhalt der Anfrage durch den Webhook muss dieser die Daten zur verarbeitenden Software zurückführen und die weitere Bearbeitung auslösen.

\subsubsection{Erstellung eines Bots}

Um ein Programm über die Telegram Bot-API anbinden zu können, muss zuvor ein Bot über Telegram erstellt werden. Hierzu wird der Bot \lstinline{@BotFather} verwendet. Sämtliche Einstellungen zu Bots werden über \lstinline{@BotFather} vorgenommen. Soll ein neuer Bot erstellt werden, werden die benötigten Informationen abgefragt. Nachdem der Anzeigename und der Benutzername festgelegt werden, erhält der Benutzer die Zugangsdaten für die HTTP-API und der Bot ist einsatzbereit. \lstinline{@BotFather} kann jederzeit erneut kontaktiert werden, um weitere Einstellungen vorzunehmen: es können Profilbild und Beschreibung geändert, sowie vordefinierte Kommandos festgelegt werden. Der Zugriff auf die HTTP-API erfolgt über die Basis-URL \lstinline{https://api.telegram.org/bot<token>/METHOD_NAME}\footnote{\url{https://core.telegram.org/bots/api\#making-requests}}, wobei \lstinline{<token>} der von \lstinline{@BotFather} erhaltene Zugriffsschlüssel und \lstinline{METHOD_NAME} die API-Methode (beispielsweise \lstinline{getUpdates} oder \lstinline{sendMessage}) ist. Weitere laut API-Dokumentation benötigte Parameter werden als Parameter der HTTP-Anfragen vom Typ 'GET' oder 'POST' übermittelt. Die Antwort der API erfolgt in Form eines JSON-Objekts.

\begin{lstlisting}[caption={Beispiel eines Aufrufs der Telegram HTTP-API. Erhalt einer Textnachricht "Hallo Welt".}, label=lst:bsp-telegram-api, numbers=none, xleftmargin=6mm]
GET https://api.telegram.org/bot123456:ABC-DEF1234ghIkl-zyx57W2v1u123ew11/getUpdates
\end{lstlisting}
\begin{lstlisting}[xleftmargin=6mm]
{
    "ok": true,
    "result": [
        {
            "update_id": 987654321,
            "message": {
                "message_id": 626,
                "from": {
                    "id": 12345678,
                    "is_bot": false,
                    "first_name": "Jonas",
                    "username": "**entfernt**",
                    "language_code": "de"
                },
                "chat": {
                    "id": 12345678,
                    "first_name": "Jonas",
                    "username": "**entfernt**",
                    "type": "private"
                },
                "date": 1662808990,
                "text": "Hallo Welt"
            }
        }
    ]
}
\end{lstlisting}

\section{Graylog Open}

Graylog Open ist ein Softwareprodukt der Firma Graylog, Inc mit dem Hauptsitz in Houston, Texas in den USA. Das Produkt stellt eine kompakte Verwaltungsoberfläche für die Erfassung von Systemprotokollen bereit, welche in einer Elasticsearch Suchmaschine vorgehalten werden. Der Quellcode der Software kann öffentlich eingesehen werden\footnote{\url{https://github.com/Graylog2/graylog2-server}}. 

Hohe Anforderungen an die (Ausfall-)sicherheit moderner und komplexer IT-Systeme führen zu der Notwendigkeit, die Funktionsfähigkeit der Systeme möglichst allumfassend und automatisiert zu prüfen. Herkömmliche Monitoringsysteme mit einem Fokus auf die Erreichbarkeit oder die Überwachung von vordefinierten Fehlerausgaben erfüllen diese Anforderungen nicht. Fehlerzustände sollen in Echtzeit, möglichst vor und spätestens zum Zeitpunkt einer durch den Anwender spürbaren Einschränkung auffallen. Informationstechnische Systeme in sämtlichen Bereichen sind längst zu komplex geworden, um alle Fehlerquellen im Vorhinein bestimmen und gezielt überwachen zu können. Aus diesem Grund wird ein anderer Ansatz als beim herkömmlichen Monitoring angewendet: die Erfassung der Systemprotokolle der zu überwachenden Systeme. Diese ermöglichen einen umfassenderen Blick auf die aktuellen Ereignisse. Durch die Anwendungsprotokolle können Fehlerzustände eines Webservers beispielsweise bereits nach dem ersten Besuch eines Besuchers einer durch den fehlerhaften Webserver beeinträchtigten Webseite erkannt werden.

\subsection{Erfassung von Systemprotokollen}
\label{sec:erfassung-systemprotokolle}

Graylog Open ist kompatibel zu einer Vielzahl von Betriebssystemen und Anwendungen. Linuxbasierte Systeme verwenden häufig einen Dienst zur zentralen und systemweiten Erfassung der Anwendungsprotokolle, welcher das in RFC 3164 standardisierte Protokoll Syslog\footnote{\url{https://www.ietf.org/rfc/rfc3164.txt}} verwendet. Dieser schreibt in der Standardkonfiguration vieler aktueller Betriebssysteme auf Basis der Linux-Distribution Debian alle Meldungen in Textform in die Datei \lstinline{/var/log/syslog}, bei Systemen auf Basis der Linux-Distribution Red Hat Enterprise Linux in die Datei \lstinline{/var/log/messages}. Die Position dieser Ausgabedatei auf dem Dateisystem ist für die Verarbeitung der Meldungen in Graylog jedoch nicht wichtig, da die Meldungen nach einer Anpassung der Konfiguration des Dienstes (offiziell wird nur die Software Rsyslog und Syslog-ng von Graylog unterstützt\footnote{\url{https://docs.graylog.org/docs/syslog}}) über das Netzwerk per UDP und mit optionaler TLS-Verschlüsselung an Graylog übertragen werden. In Graylog muss hierzu die Annahme von Daten über das Netzwerk mit dem syslog-Protokoll aktiviert werden. Mittels einer Input-Konfiguration wird ein Port an der zum überwachenden System nächstgelegenen Netzwerkschnittstelle eröffnet, auf welchem die Software auf eintreffende Meldungen im syslog-Format lauscht.

Auch mit Docker bereitgestellte Microservices können global überwacht werden, ohne die Konfiguration der Container oder sogar die der Anwendungen in einem gestarteten Container einzeln anpassen zu müssen. Hierzu wird nicht das syslog-Protokoll, sondern das von Docker ohnehin unterstützte\footnote{\url{https://docs.docker.com/config/containers/logging/gelf/}} Protokoll GELF verwendet. Es ist eine zentrale Änderung der Konfiguration des Docker-Dienstes notwendig, um fortan die Protokolle aller neu gestarteten Container über das wahlweise UDP- oder TCP-basierte GELF-Protokoll an die Graylog-Instanz zu senden. Die an Graylog übertragenen Daten entsprechen der Ausgabe des Kommandos \lstinline{docker logs}. Um den Empfang von Daten über das GELF-Protokoll seitens Graylog zu ermöglichen, ist es analog zur Einrichtung für das syslog-Protokoll notwendig, mittels einer Input-Konfiguration einen Port auf einer Netzwerkschnittstelle zu reservieren.

Systemprotokolle aus Windows können nicht direkt verarbeitet werden. Es ist der Einsatz einer Middleware wie Winlogbeat notwendig, welche die Protokolle auf dem System erfasst und die Daten über ein Netzwerkprotokoll an Graylog sendet.\footnote{\url{https://docs.graylog.org/docs/windows}}

\subsection{Verarbeitung}

Für die Verarbeitung der erfassten Daten stellt Graylog dem Systemadministrator eine Vielzahl von Möglichkeiten zur Verfügung. Hier werden lediglich die Möglichkeiten erläutert, welche für das Ziel der Abschlussarbeit, also der sprachgesteuerten Verarbeitung durch den Telegram-Bot, relevant sind. Graylog erhält die Daten verschiedener Systeme über die konfigurierten Input-Konfigurationen (vgl. \autoref{sec:erfassung-systemprotokolle}). Die empfangenen Daten werden in einer Elasticsearch Suchmaschine hinterlegt und nach vom Administrator definierten Filterausdrücken durchsucht. Hierbei können bestimmte Parameter mit regulären Ausdrücken aus den empfangenen Daten in eigenen Attributen hinterlegt werden, wie in \autoref{fig:graylog-bsp-msg} zu sehen. Die erfassten Nachrichten sind im Anschluss inkl. der oben genannten Attribute über die Weboberfläche und die REST-API durchsuchbar. Für die Suche wird eine an Apache Lucene (Programmbibliothek für Volltextsuche) angelehnte Syntax verwendet. Diese ermöglicht die Entwicklung komplexer Suchbegriffe. Ein möglicher Suchbegriff für den ganzzahligen Parameter \lstinline{http_response_code} aus der \autoref{fig:graylog-bsp-msg} im Bereich 400 bis 499 wäre beispielsweise \lstinline{http_response_code:[400 TO 499]}. 

\begin{figure}[h!]
\centering
\includegraphics[scale=0.7]{graylog-bsp-msg}
\caption{Extrahieren von Daten aus eingehenden Meldungen in Graylog.}
\label{fig:graylog-bsp-msg}
\end{figure}

\subsection{Zugriff per API}

Graylog stellt eine REST-API zur Verfügung, auf welche sowohl mit der integrierten Webschnittstelle als auch programmgesteuert aus der Ferne zugegriffen werden kann. Die integrierte Webschnittstelle nutzt die API für das Beziehen der Informationen aus der Elasticsearch Suchmaschine. Jede Graylog-Instanz bietet standardmäßig eine interaktive Oberfläche an, über welche auf die Dokumentation der verfügbaren Funktionen zugegriffen werden kann. Es ist ebenfalls möglich, die API-Funktionen aus dem Browser heraus mit selbstgewählten Parametern aufzurufen. Dies erleichtert die Softwareentwicklung, da so keine externen Werkzeuge für den Zugriff auf die API verwendet werden müssen.

\section{Amazon Web Services}

Der Cloudanbieter AWS ist ein Tochterunternehmen des US-amerikanischen Versandhändlers Amazon. Das Unternehmen bietet seine Dienste vor allem für professionelle Endanwender und Unternehmen an. AWS bietet eine Vielzahl von Diensten, welche on-demand (dt. auf Abruf) verfügbar sind, nach dem Prinzip 'pay as you go' (dt. 'abgerechnet nach Verbrauch') berechnet werden. Zu den bekanntesten Diensten zählen EC2 (elastic compute cloud), welcher virtuelle Maschinen bereitstellt, sowie S3 (simple storage service), für das Buchen von Speicherkapazität. AWS gehört neben den allesamt US-amerikanischen Anbietern Microsoft Azure, Google Cloud, Oracle Cloud sowie dem chinesischen Anbieter Alibaba Cloud laut dem Marktforschungsunternehmen Gartner zu den führenden Cloudanbietern weltweit\footnote{\url{https://www.gartner.com/technology/media-products/reprints/AWS/1-271W1OSP-DEU.html}}. Derzeit gibt es keinen gleichwertigen Anbieter aus Europa. IONOS und Strato aus Deutschland sowie OVH aus Frankreich bieten nur einen Bruchteil der Umfänge der Konkurrenten an.

AWS betreibt weltweit zahlreiche Rechenzentren weltweit und stellt einen Großteil der Dienste an allen Standorten zur Verfügung. Zahlreiche Dienste lassen sich über die Webschnittstelle 'AWS Konsole' verwalten. Als on-demand Anbieter steht für sämtliche Dienste auch die Bedienung über eine API-Softwareschnittstelle zur Verfügung, sodass Ressourcen bei Bedarf von Programmen selbstständig gebucht und vollständig verwaltet werden können. Die Abrechnung variiert je nach Dienst. Rechenressourcen werden nach Stunden oder Minuten abgerechnet, Speicherressourcen nach Datenmenge und auftragsbasierte Dienste (vgl. \autoref{sec:amazon-transcribe}) nach Auftragskontingenten. AWS bietet zusätzlich ein freies Kontingent an. Die anfallenden Kosten lassen sich im Vorhinein mit dem AWS Pricing Calculator\footnote{\url{https://calculator.aws/}} bestimmen. 

Für die Verwendung der AWS API mit der Programmiersprache Python stellt ein Entwicklerteam das offizielle SDK unter dem Namen Boto3 zur Verfügung. Das SDK erleichtert die Bedienung der verschiedenen API-Funktionen in Verbindung mit der Programmiersprache Python. Es wird mit von einzelnen API-Anfragen entkoppelten Objekten gearbeitet\footnote{\url{https://boto3.amazonaws.com/v1/documentation/api/latest/guide/resources.html}}, damit entfällt die eigenständige Kommunikation mit der API durch die implementierte Software (beispielsweise mit der Programmbibliothek Requests) vollständig. Das SDK übernimmt weiterhin die Authentifizierung gegenüber der API. Ein Objekt repräsentiert jeweils einen AWS-Dienst (siehe folgende Unterabschnitte) in der Softwareumgebung und ermöglicht eine einfache Einbindung in die Software. Die verfügbaren API-Funktionen können als Methoden des Objekts in Python direkt verwendet werden.

\subsection{Amazon Transcribe}
\label{sec:amazon-transcribe}

Mit dem Dienst Amazon Transcribe können Audiodateien in Text mittels künstlicher Intelligenz transkribiert werden. Der Dienst unterstützt mehrere Sprachen und Dialekte. Standardmäßig wird ein allgemeines Sprachmodell des Anbieters verwendet. Es ist auch möglich, ein eigenes Sprachmodell zu trainieren und importieren. Bei der Bedienung über die AWS Konsole muss ein URI zu einer Audiodatei in einem S3-Speicher angegeben werden. Im Bezug auf S3-Speicher unterscheiden sich URI und URL voneinander. Die Unterschiede werden im \hyperref[sec:unterschied-uri-url]{nächsten Unterkapitel} erläutert. Der Dienst hinterlegt den Text in einer Datei ebenfalls in einem S3-Speicher, optional kann dabei der Quellspeicher eingestellt werden. 

Eine Übersetzung in Echtzeit ist nicht möglich. Das Limit für gleichzeitige Vorgänge pro Benutzer liegt bei 100. Neue Aufträge können so lange nicht eingereicht werden, bis die Anzahl wieder unter dem Limit liegt. Bei einer absehbaren Überschreitung des Limits kann eine Auftragswarteschlange verwendet werden, um die Aufträge zuzuführen. 

Die Abrechnung erfolgt außerhalb des freien Kontingents nach Zeitkontingenten und beginnt bei 0,024 USD pro angefangene Minute.

\subsubsection{Unterschied von URI und URL}
\label{sec:unterschied-uri-url}

Ein URL (engl. 'uniform resource locator') gibt den Weg zu einer Ressource an. Ein URI (engl. 'uniform resource identifier') gibt nicht unbedingt den Weg zu einer Ressource an, aber dessen Identifikationsmerkmal (beispielsweise ein vergebener Name). S3-URIs enthalten den Dateipfad einer Datei in einem S3-Speicher in jeweils einer geographischen Region (Beispielsweise am AWS Standort 'eu-central-1' in Frankfurt). In anderen Regionen können gleichnamige S3-Speicher erstellt werden. Um zwischen diesen Speichern zu unterscheiden, wird ein URL benötigt, da dieser den öffentlichen Ort des Dokuments angibt. Ein Beispiel für eine S3-URI ist \lstinline{s3://ba-telegram-graylog-bot/audio/tts001.mp3}. Ein Beispiel für eine S3-URL zu derselben Ressource am Standort 'eu-central-1' ist \lstinline{https://ba-telegram-graylog-bot.s3.eu-central-1.amazonaws.com/audio/tts001.mp3}. 

\subsection{Amazon Polly}

Polly ist ein TTS-Dienst (Abkürzung für 'text-to-speech', dt. 'Text zu Sprache') und bildet das Gegenstück zu Amazon Transcribe. Der Dienst unterstützt ebenfalls mehrere Sprachen und Dialekte. Je nach Sprache können verschiedene Modelle verwendet werden, welche verschiedene Persönlichkeiten und Geschlechter darstellen. Dabei wird zwischen den Typen 'standard' und 'neural' unterschieden. 'Neural'-Modelle erzeugen eine gegenüber dem Typ 'standard' optimierte Ausgabe, welche der menschlichen Aussprache so ähnlich wie möglich kommen soll. 

Die Abrechnung erfolgt außerhalb des freien Kontingents nach Zeichenkontingenten und beginnt bei 4 USD für eine Million Zeichen. Der Typ 'neural' kostet bei gleicher Verwendung etwa das Vierfache. 

\section{Verwandte Arbeiten}

Im Folgenden werden wissenschaftliche Arbeiten mit ähnlichen Zielsetzungen vorgestellt und die Unterschiede zu dieser Arbeit erläutert.

\subsection{Voice-controlled order system}

Die Abschlussarbeit von David Höijer und Hannes Jansson von der staatlichen Hochschule Halmstad in Schweden mit dem Titel "Voice-controlled order system" und veröffentlicht am 14.06.2021 behandelt die Planung und Implementierung eines Sprachassistenten für die Aufgabe von Bestellungen bei Lieferdiensten für Lebensmittel\footnote{\url{http://urn.kb.se/resolve?urn=urn:nbn:se:hh:diva-45033}}. Für die Spracherkennung und Verarbeitung der eingegangenen Aufträge wird auf Dienste der Cloudanbieter Amazon Webservices und Google Cloud zurückgegriffen. Die Unterschiede in der Umsetzung zur vorliegenden Arbeit bestehen darin, dass David Höijer und Hannes Jansson sich für eine Spracherkennung mit künstlicher Intelligenz durch Google Dialogflow und damit gegen ein statisches Modell, wie es in dieser Arbeit eingesetzt wird, entschieden haben. 

\subsection{Chatbot capable of information search}

Michal Ďurista entwickelte im Rahmen seiner Bachelorarbeit an der technischen Universität Brünn in Tschechien aus dem Jahr 2019 einen Chatbot, welcher Benutzern Informationen von einer Webseite auf Abruf zur Verfügung stellt\footnote{\url{https://www.fit.vut.cz/study/thesis/21921/}}. Die Benutzer kontaktieren den Bot dabei in Alltagssprache. Mithilfe von künstlicher Intelligenz analysiert der Bot die Nachrichtenbestandteile und ermittelt so die gewünschten Informationen. Der Autor erwähnt in der Arbeit Aspekte der Verarbeitung natürlicher Sprache und verwendet für die Implementierung Dienste von Microsoft Azure. Der Unterschied zur vorliegenden Arbeit besteht darin, dass die Informationen nicht maschinenlesbar vorliegen. Die Software bestimmt zentrale Schlüsselwörter der Anfragen und verwendet diese für eine oder mehrere Volltextsuchen über die vorliegenden Informationen. Die Auskünfte des Bots beschränken sich dabei auf Produktinformationen zu technischen Bauteilen eines einzelnen Unternehmens.

\chapter{Planung}

In diesem Kapitel wird die Planung der Software mit Prinzipien des Softwareengineerings erläutert. 

\section{Auswahl der Programmiersprache}

Die Software soll in der Skriptsprache Python entwickelt werden. Die Programmiersprache bietet eine einmalige Kombination von Vorteilen gegenüber anderen populären Programmiersprachen: die Plattformunabhängigkeit wird durch die Verfügbarkeit von Interpretern sowohl für unixoide Betriebssysteme als auch Windows ermöglicht. Die Software liegt jederzeit im Quellcode vor und kann so einfach gewartet und erweitert werden. Des weiteren existiert aufgrund der hohen Popularität \footnote{https://www.tiobe.com/tiobe-index/python/} eine reichhaltige Auswahl an Bibliotheken und Onlinedokumentationen, welche die Entwicklung der Software vereinfachen. 

\section{Anforderungen}

\subsection{Systemumgebung}

Die Software soll auf den nicht interaktiven Betrieb als Dienst auf unixoiden Betriebssystemen ausgelegt werden. Dies hat u.A. Auswirkungen auf die geplanten Benutzerschnittstellen. Das Programm soll zur Laufzeit keine Konsoleneingaben verlangen, da diese nur bei einem interaktiven Betrieb, z.B. in der Shell, vorgenommen werden können. Stattdessen soll mit dem Anwender vollständig über den Telegram-Messenger interagiert werden und administrative Einstellungen sollen über Konfigurationsdateien vorgenommen werden können. Zur Anwendung von geänderten Einstellungen ist u.U. ein Neustart der Software notwendig. Die Software sollte daher zustandslos arbeiten, um einen möglichen Verlust von eingegangenen und noch nicht verarbeiteten Daten zu verhindern. Weiterhin sollte die durch einen Neustart verursachte Ausfallzeit so gering wie möglich gehalten werden. Die Software soll Ereignismeldungen über die Standardausgabe sowie über eine textbasierte Protokolldatei ausgeben, um Fehleranalysen zu vereinfachen.

\subsection{Systemkonfiguration}
Für den Betrieb der Software ist eine vorinstallierte Python 3 Umgebung notwendig. Die benötigten Bibliotheken sollen über den Paketmanager pip bezogen und aktuell gehalten werden können. Auch wenn die Software auf den Betrieb als Dienst ausgelegt werden soll, soll eine interaktive Verwendung sowohl mit unixoiden Betriebssystemen als auch mit Windows möglich sein. Weiterhin muss eine Verbindung zum Internet über die Protokolle HTTP und HTTPS möglich sein, um mit externen Diensten kommunizieren zu können.

\section{Programmablauf}

\subsection{Grundsätzlicher Aufbau}

Nach dem Start der Software muss geprüft werden, ob ein einwandfreier Betrieb möglich ist. Dazu ist es notwendig, die Erreichbarkeit und Funktion sämtlicher Dienste mittels geeigneter API-Abfragen zu prüfen. Nachdem die Vorbereitungen abgeschlossen sind, kann in den Regelbetrieb übergegangen werden, in welchem sich das Programm bis zum Programmende befindet. 

\subsection{Regelbetrieb}

Im Regelbetrieb reagiert die Software in Echtzeit auf eingehende Nachrichten vom Anwender. 
Um den Betrieb der Software auch mit nicht-öffentlichen IP-Adressen (beispielsweise in einem Heimnetzwerk hinter einem NAT-Router oder einer Firewall) zu ermöglichen, soll das Verfahren des \lstinline{long-polling} für den Abruf von Informationen von der Telegram API verwendet werden (vgl. \autoref{sec:telegram-getting-updates}).

Trifft eine Nachricht ein, muss wie folgt verfahren werden: 

\begin{enumerate}
\item Benutzerautorisierung prüfen (der Bot ist öffentlich erreichbar, eine Zugriffsbeschränkung seitens Telegram ist nicht vorgesehen)
\item Prüfung der Nachrichtenbestandteile
\subitem 2.1 Bei einer Sprachnachricht: Transkribieren der Audiodatei in Text
\item Erfassung der für eine Abfrage notwendigen Informationen
\item Stellen der Abfrage an das Graylog-Backend
\item Erhalten einer Antwort vom Graylog-Backend
\item Einbetten der erhaltenen Rohdaten in einen Nachrichtentext
\item Umwandeln des Texts in eine Audiodatei mittels Text-To-Speech
\item Senden einer Antwort an den Absender der Nachricht. Audiodatei als Sprachnachricht der Nachricht anhängen.
\end{enumerate}

\subsection{Spracherkennung}

Bei der Ausarbeitung eines Konzepts für die Funktionsweise der Spracherkennung müssen einige funktionale und nicht-funktionale Eigenschaften beachtet werden. Die Entscheidung wird u.A. beeinflusst durch Aspekte der... 

\begin{itemize}
\item Erweiterbarkeit: es soll einfach und insbesondere ohne ein notwendiges Training von Sprachmodellen möglich sein, neue Systeme und Abfragen zu definieren.
\item Fehlertoleranz: gesprochene Sprache enthält Umgangssprache und verändert sich durch grammatikalische Eigenschaften je nach Satzbau leicht in der Aussprache. Die Erkennung muss trotz dieser Veränderungen zuverlässig funktionieren.
\item Rechenkapazität: die Geschwindigkeit (Wartezeit während dem Vorgang) der Spracherkennung sollte nicht von limitierten Rechenressourcen oder der Länge der zu übersetzenden Sprache abhängen.
\item finanziellen Kosten: diese sollten möglichst gering gehalten werden.
\item Plattformunabhängigkeit: die Software und ggf. trainierte Modelle sollten wie die gewählte Programmiersprache Python plattformunabhängig und portabel sein.
\item Nachvollziehbarkeit: während der Entwicklung und bei der Definition neuer Begriffe sollten auftretende Fehler einfach erkenn- und behebbar sein.
\item Sprache: das System soll Sätze verstehen, welche Begriffe aus mehreren Sprachen (deutsch und englisch) beinhalten.
\end{itemize}

Es bestehen verschiedene Möglichkeiten, Sprache zu Text zu transkribieren und den Inhalt im Anschluss zu analysieren, um das Anliegen des Anwenders zu erkennen und eine passende Anfrage für die Suchmaschine in Graylog zu bilden. Für die Transkription ist es notwendig, künstliche Intelligenz einzusetzen. Diese kann lokal oder entfernt ausgeführt werden. Die entfernte Ausführung bietet Vorteile bezüglich der Plattformunabhängigkeit und der (von den lokalen Ressourcen unabhängigen) Geschwindigkeit. Es stehen verschiedene gleichwertige Online-Dienste für die Transkription zur Verfügung, darunter "Watson Speech to Text" von IBM und "Amazon Transcribe" von AWS.

\subsubsection{Syntax}

Liegt die eingegangene Nachricht als Text vor, müssen die Inhalte analysiert werden. Aufgrund des notwendigen Trainings für neue Begriffsdefinitionen ist der Einsatz von künstlicher Intelligenz bei dieser Programmkomponente nicht sinnvoll. Der Aufbau der Nachricht muss stattdessen einem festen Muster folgen, welches durch eine Prozedur analysiert werden kann. Hierzu eignet sich die Verwendung von Schlüsselwörtern. Diese bieten den Vorteil, dass keine Analyse der Grammatik notwendig ist. Um die Erweiterung zu vereinfachen, wird ein Aufbau aus Produktkategorie, Eigenschaft und Zeitraum gewählt. Die Produktkategorie entspricht der abzufragenden Gerätegruppe, beispielsweise \lstinline{Webserver}. Für jede Produktkategorie können Eigenschaften definiert werden, für die Abfrage von Ereignissen mit einem Statuscode \lstinline{4xx} oder \lstinline{5xx} der Gruppe \lstinline{Webserver} beispielsweise \lstinline{Fehlermeldungen}. Ein weiteres Schlüsselwort führt zu den Informationen für den abzufragenden Zeitraum, welcher relativ angegeben wird (\lstinline{letzte fünf Tage}, \lstinline{letzte Woche}, \lstinline{letzte 20 Minuten}). Schließlich muss die Erkennung von Zahlwörtern zuverlässig funktionieren.

\chapter{Implementierung}

\section{Grundlegender Aufbau}

Um eine modulare Implementierung der Software zu ermöglichen und darüber hinaus den Quellcode öffentlich bereitstellen zu können, ist es notwendig, die Softwareentwicklungsumgebung entsprechend zu gestalten:

Um die Installation des Programms zu vereinfachen, befindet sich eine Liste mit den Abhängigkeiten zu Bibliotheken von Drittanbietern in der Datei \lstinline{requirements.txt}. Diese Datei kann verwendet werden, um alle für die Ausführung benötigten Abhängigkeiten über den Paketmanager \lstinline{pip} aus dem Python Package Index \lstinline{PyPI} zu laden. Die Datei beinhaltet ebenfalls Angaben zu benötigten Softwareversionen.

Die Programmkonfiguration wird über die Datei \lstinline{constants.py} vorgenommen. Diese beinhaltet Variablen, mit welchen der Ablauf des Programms gesteuert werden kann (beispielsweise können detaillierte Meldungen zum Programmablauf aktiviert oder bestimmte Telegram-Benutzer für den Zugriff auf den Bot autorisiert werden). Des Weiteren beinhaltet die Datei Variablen, welche zur Laufzeit mit Objektdefinitionen der boto3-Bibliothek überschrieben werden. Dies ermöglicht einen durchgehenden Zugriff auf die AWS-API ohne wiederkehrende Authentifizierungen.

Sämtliche Zugangsdaten zu Telegram, Graylog und AWS werden in der Datei \lstinline{.env} hinterlegt. Die Datei wird nicht in das öffentliche GitHub-Repository hochgeladen.

\chapter{Praxiseinsatz}
\label{cha:praxiseinsatz}

Dieses Kapitel beschreibt den praktischen Einsatz der Software und beleuchtet zudem einen bisher nicht diskutierten Aspekt der IT-Sicherheit.

\section{Anwendung}

Der Code der entwickelten Software wurde über ein GitHub-Repository\footnote{\url{https://github.com/Jomibe/ba}} veröffentlicht, sodass die Installation und Konfiguration des Bots für eine eigene Installation von Graylog wiederverwendet oder optimiert werden kann. Für den Betrieb ist es notwendig, zuvor die angeschlossenen Dienste inkl. der API-Zugriffsdaten zu konfigurieren. Im Anschluss kann die Software mit einem Python Interpreter ausgeführt werden und ist einsatzbereit, sobald die Vorbereitungsphase abgeschlossen ist. Mithilfe der aktivierten ausführlichen Ausgaben zum Programmablauf können die Vorgänge und dessen Auslöser gut zurückverfolgt werden. Im Folgenden zeigen zwei Abbildungen die Interaktion mit dem Bot über den Telegram Messenger:

\begin{figure}[h!]
    \centering
    \begin{minipage}{0.5\textwidth}
        \raggedright
        \includegraphics[width=0.9\textwidth]{bsp-betrieb}
        \caption{Bot beantwortet eine \\\hspace{\textwidth}Anfrage in Telegram.}
        \label{fig:bsp-betrieb}
    \end{minipage}\hfill
    \begin{minipage}{0.5\textwidth}
        \raggedleft
        \includegraphics[width=0.9\textwidth]{bsp-fehler}
        \caption{Bot reagiert auf fehlerhafte \\\hspace{\textwidth}Anfrage in Telegram.}
        \label{fig:bsp-fehler}
    \end{minipage}
\end{figure}

Die \autoref{fig:bsp-betrieb} zeigt, wie der Bot eine Anfrage beantwortet. Zuerst wird dem Benutzer eine Rückmeldung gesendet, sobald die Transkription fertiggestellt ist. Im Anschluss erfolgt die Beantwortung der eingesendeten Anfrage. Alle Nachrichten des Bots werden in einer kombinierten Sprachnachricht gesendet, welche zusätzlich den gesprochenen Text in geschriebener Form enthält. Zusätzlich ist es möglich, Anfragen per Textnachricht einzusenden, wie \autoref{fig:bsp-fehler} zeigt:

In \autoref{fig:bsp-fehler} wurde dem Bot eine Nachricht zugestellt, welche nicht die erforderlichen Informationen beinhaltet. Der Bot prüft dies anhand der konfigurierten Schlüsselwörter und weißt den Anwender auf jede fehlende Information hin. 

\section{Qualitätskontrolle}

Nach Abschluss der Implementierung soll die Qualität der Software bezüglich der Benutzbarkeit im geplanten Anwendungsfall kurz erläutert werden. Die Qualität der Anwendung hängt stark von der Qualität der Spracherkennung ab, da hiervon die höchste Fehlerwahrscheinlichkeit ausgeht. Mit dem Vergleich verschiedener Dienste (vgl. \autoref{sec:vergleich-transkrip}) konnte dieses Qualitätsmerkmal bereits vor der Implementierung optimiert werden. Durch die Möglichkeit, Anfragen ebenfalls als Textnachricht an die Software zu übermitteln, kann der Anwender die Fehlerwahrscheinlichkeit bei der Erkennung des Anliegens reduzieren. Weiterhin ist es möglich, lokale Transkriptionsdienste, beispielsweise von der Tastatur-App auf einem Mobiltelefon zu nutzen. Ein weiteres Werkzeug für die Verbesserung der Spracherkennung sind die Aliasdefinitionen. Der Vergleich der verschiedenen Dienste hat bereits bei einer kleinen Stichprobe gezeigt, dass die Spracherkennung bei Fachbegriffen nicht verlässlich genug arbeitet. Die Definition von Begriffen aus der Alltagssprache als Alias für Fachbegriffe führt daher zu einer Verbesserung der Erkennungsquote und somit zu einem gesteigerten Qualitätsempfinden der Anwendung aus Sicht der Benutzer.

\section{IT-Sicherheit}

Aus Sicht der IT-Sicherheit bringt die Verwendung des Telegram-Bots Vorteile gegenüber der Verwendung der Graylog-Webschnittstelle, sofern der Zugriff von außerhalb des Netzwerks erfolgt. Die Graylog-Instanz sammelt die Systemprotokolle sämtlicher Geräte im Netzwerk und ist damit ein besonders schützenswertes System in einem Firmennetzwerk. Der Zugriff von außerhalb sollte gut abgesichert werden. Es gibt mehrere Möglichkeiten, um webbasierte Systeme in einem internen Netzwerk über das Internet erreichbar zu machen. 

Eine simple Möglichkeit besteht darin, auf dem Internetrouter eine Portweiterleitung einzurichten. Besser ist jedoch eine Veröffentlichung über einen 'Reverse Proxy'. Ein Proxy ist ein System, welches stellvertretend für ein weiteres System eine Kommunikation mit dem gewünschten Partner übernimmt. Häufig werden Proxys in Firmennetzwerken für die Filterung von besuchten Internetseiten eingesetzt. Dabei wendet sich ein PC im internen Netzwerk an den Proxyserver und fragt eine Internetressource an. Der Proxy agiert als Vermittler und bezieht die Internetressource, bevor er diese an den PC weiterleitet. Bei einem Reverse Proxy ist das Prinzip umgekehrt. Hier übernimmt der Vermittler die Kommunikation von Geräten außerhalb eines Netzwerks (beispielsweise dem Internet) stellvertretend für die Quelle und kontaktiert den gewünschten Verbindungspartner im internen Netzwerk. Der Vorteil eines Reverse Proxys gegenüber einer Portweiterleitung besteht darin, dass die HTTP-Anfragen detailliert bearbeitet, gefiltert und protokolliert werden können. Bestimmte Adressmuster, welche für Angriffe verwendet werden, können so beispielsweise direkt blockiert werden. Der alleinige Einsatz eines Reverse Proxys mit Filter für die Veröffentlichung über das Internet reicht bei einem schützenswerten System nicht aus. Daher wird häufig eine HTTP-Authentifizierung zwischengeschaltet, welche durch die Anforderung von vom Ziel unabhängigen Zugangsdaten eine zusätzliche Sicherheitsebene schafft. Ein Reverse Proxy mit auf die Zielapplikation abgestimmten Filtermechanismen sowie einer zusätzlichen Authentifizierungsschicht bietet ausreichenden Schutz für eine Veröffentlichung im Internet. Der Einsatz eines solchen Reverse Proxies mit Graylog ist jedoch nicht möglich, da Graylog nicht kompatibel zu Anfragen mit HTTP-Authentifizierung ist\footnote{Fehlerbericht und Erweiterungsvorschlag eines Anwenders, welcher Graylog über einen Reverse Proxy mit HTTP-Authentifizierung erreichbar machen möchte: \url{https://github.com/Graylog2/graylog2-server/issues/6831}}. Eine Alternative zum Reverse Proxy bietet ein Dienst, welcher die Funktion eines Proxys oder Stellvertreters für die Verbindung zur Graylog-Webschnittstelle schafft. Dieser Dienst kann für beliebige Funktionen implementiert werden, welche über die Graylog REST-API Daten beziehen und weiterverarbeiten. Die in dieser Abschlussarbeit implementierte Software ist ein Beispiel für einen solchen Vermittler auf Applikationsebene \cite[S. 12 ff.]{bsi-websec}.

\chapter{Fazit}
\label{cha:fazit}

\section{Zusammenfassung}

Das Ziel der Abschlussarbeit war die Implementierung einer Software, mit welcher Systemadministratoren sich in Echtzeit über den Status eines administrierten Netzwerkes informieren können. Die Informationsquelle für Abfragen ist die Software Graylog Open, welche Systemprotokolle sämtlicher Geräte in einem Netzwerk zentral sammelt und durchsuchbar macht. Den Administratoren sollte es ermöglicht werden, über Sprachnachrichten Anfragen zu verfassen. Die Aufgabe der zu implementierenden Software ist es, die Informationen aus der über den Messenger Telegram gesendeten Sprachnachricht zu extrahieren und damit eine Anfrage an Graylog zu stellen. Schließlich sollte die Software die Antwort von Graylog wiederum als Sprachnachricht an den Benutzer zurücksenden. 

Die Implementierung der Software verlief planmäßig, sodass die gestellten Anforderungen erfüllt werden konnten. Während der Entwicklung und bei der Prüfung erster Prototypen fiel auf, dass die nicht-funktionalen Anforderungen Verarbeitungsgeschwindigkeit und Erkennungsgenauigkeit der Spracherkennung an ein Produkt, welches ausschließlich per Sprache bedient werden soll hohe Ansprüche an die Qualität der Implementierung stellen. Durch die Verwendung von asynchronen Funktionen konnte die Verarbeitungsgeschwindigkeit erheblich reduziert werden. Eine weitere Funktion, welche zur Verbesserung der Erkennungsgenauigkeit beigetragen hat, waren die implementierten Aliasdefinitionen. Mit diesen konnte die im Gegensatz zur Transkription von alltäglicher Sprache niedrige Treffergenauigkeit von Fachbegriffen aus dem Bereich der Informatik gut kompensiert werden. 

\section{Ausblick}

Es sind einige Erweiterungen der Funktionalität der entwickelten Software denkbar. 

\subsection{Nebenläufigkeit}

Die ersten Voraussetzungen für eine vollständige Nebenläufigkeit von Prozessen wurden bereits mit der Verwendung des Amazon Transcribe SDK erfüllt. Derzeit verarbeitet der Bot die eingehenden Nachrichten synchron, bzw. nacheinander. Es ist jederzeit möglich, neue Nachrichten einzusenden, diese werden jedoch sequenziell statt parallel verarbeitet. Um die parallele Verarbeitung von eingehenden Nachrichten zu ermöglichen, ist es notwendig, asynchrone Funktionen auch in weiteren Programmmodulen einzusetzen. Weiterhin ist es denkbar, die Software für den Betrieb mit mehreren Threads anzupassen. Anfragen würden dann von einer Softwarekomponente angenommen und von einem jeweils dafür erzeugten Thread bearbeitet werden, welche das Ergebnis asynchron an die Hauptinstanz zurückführen. 

\subsection{Datenaufbereitung}

Derzeit gibt die Software für jede Anfrage die Anzahl der erfassten Ereignisse zurück. Diese Angabe ist nicht für jede Abfrage sinnvoll. Zukünftig könnten weitere Funktionen für die Aufbereitung der Informationen inkl. weiteren Schlüsselwörtern implementiert werden, welche beispielsweise den Minimal- oder Maximalwert oder in einer Umgebung mit Webservern die fünf häufigsten Statuscodes zurückgeben.

\subsection{Verwaltung per Telegram}

Änderungen am Verhalten der Software und insbesondere an den Zuordnungen der Suchbegriffe zu den Schlüsselwörtern können derzeit nur über eine Anpassung von Konfigurationsdateien vorgenommen werden. In Zukunft könnte ein Rollenkonzept eingeführt werden, welches es bestimmten Benutzern erlaubt, Änderungen an den Zuordnungen vornehmen zu dürfen und neue Abfragen definieren zu können.

\input{70_anhang}

\backmatter
\printbibliography[title=Literaturverzeichnis]
\begingroup
\let\clearpage\relax
\listoffigures
\listoftables
\lstlistoflistings
\endgroup
\end{document}
