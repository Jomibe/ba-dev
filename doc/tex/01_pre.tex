\begin{titlepage}
\begin{center}
\begin{center}
\includegraphics{01_Logo-CMYK}
\end{center}

\vspace*{10mm}
\huge
\textbf{\titeldeutsch}

\vspace{10mm}
\Large
\titelenglisch

\vspace{15mm}
\LARGE
\textsc{\abschlussarbeit}

\vspace{20mm}
\large
\name

\hochschule

\datum
\end{center}
\end{titlepage}

\clearpage

\normalsize\normalfont

\thispagestyle{plain}
\begin{tabular}{ll}
Autor: & \name \\
Referent: & \erstpruefer \\
Korreferent: & \zweitpruefer \\
Eingereicht: & \datum
\end{tabular}

\chapter*{Zusammenfassung}
\addcontentsline{toc}{chapter}{Zusammenfassung}

Das Thema dieser Abschlussarbeit ist die Entwicklung eines virtuellen Assistenten (Bot) für den Telegram Messenger, welcher Netzwerkadministratoren bei ihrer täglichen Arbeit durch die Möglichkeit unterstützen soll, in Echtzeit Informationen zu ihrem verwalteten Netzwerk abzurufen. Dabei wird auf die Software Graylog zurückgegriffen, welche die zentrale Verwaltung von anfallenden Systemprotokollen für ein Netzwerk übernimmt. Der Systemadministrator kann vordefinierte Abfragen in der Bot-Software hinterlegen und verknüpfen, sodass diese auf Anfrage jederzeit an die in Graylog integrierte Suchmaschine übermittelt werden können. Die Kommunikation zwischen Bot und Anwender erfolgt über Sprachnachrichten. Um die Nachrichten verarbeiten zu können, wandelt der Bot eingehende Sprachnachrichten mit einer Spracherkennung in Text um.

\chapter*{Erklärung}

Ich erkläre hiermit, dass ich die vorliegende Arbeit selbstständig verfasst und dabei keine anderen als die angegebenen Hilfsmittel benutzt habe. Sämtliche Stellen der Arbeit, die im Wortlaut oder dem Sinn nach Werken anderer Autoren entnommen sind, habe ich als solche kenntlich gemacht. Die Arbeit wurde bisher weder gesamt noch in Teilen einer anderen Prüfungsbehörde vorgelegt und auch noch nicht veröffentlicht.

\bigskip
\noindent
\datum

\vspace{25mm}

\noindent
\name
