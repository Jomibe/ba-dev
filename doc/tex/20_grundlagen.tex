\chapter{Grundlagen}

In diesem Kapitel werden die für die Entwicklung benötigten verwendeten technischen Grundlagen erläutert.

\section{REST}

REST ist eine Abkürzung für Representational State Transfer und wurde von Roy Fielding in seiner Dissertation aus dem Jahr 2000 \cite{rest} definiert. Es handelt sich hierbei um eine Auflistung von Eigenschaften und Anforderungen, welche durch eine REST-API erfüllt werden sollen. REST-APIs werden für viele Dienste des Internets meist unsichtbar für die Anwender für die Kommunikation zwischen informationstechnischen Systemen verwendet. In der englischen Sprache wird häufig der Begriff "RESTful" für Ressourcen verwendet, welche die REST-Prinzipien anwenden.

\section{Telegram Messenger}

Der Dienst Telegram bietet seinen Nutzern eine kostenlose Möglichkeit um miteinander u.A. über Apps für die Mobilbetriebssysteme Android und iOS zu kommunizieren. Telegram hatte nach nach eigenen Angaben weltweit im Juni 2022 erstmals über 700 Millionen monatliche Nutzer und befand sich unter den fünf Apps mit den höchsten Downloadzahlen \cite{telegram-700-million}. 

\subsection{Beziehen von Aktualisierungen von der Telegram API}

Laut der technischen Dokumentation der Telegram API \cite{telegram-getting-updates} stehen zwei Möglichkeiten zur Verfügung, um Aktualisierungen zu erhalten: \lstinline{long-polling} der API-Methode \lstinline{getUpdates()} oder die Verwendung eines \lstinline{webhooks}. Die beiden Möglichkeiten verwenden unterschiedliche Konzepte und bieten damit verschiedene Vor- und Nachteile: im Fachbereich der Informatik steht der Begriff \lstinline{polling} für eine dauerhafte wiederkehrende Abfrage von Informationen bei einem Dienst. Diese Technik beansprucht viel CPU-Zeit eines Systems und wird aufgrund des geringen Gegenwerts der meist leer ausgehenden Abfragen als teuer bezeichnet. \lstinline{long-polling} beschreibt eine Technik, bei welcher der Client einer HTTP-Abfrage einen verlängerten Zeitraum bis zum Erhalt der Antwort vom Server akzeptiert, ohne die Abfrage aufgrund einer Zeitüberschreitung abzubrechen. Die Länge des Zeitraums ist variabel. \lstinline{long-polling} bietet damit gegenüber dem klassischen \lstinline{polling} den Vorteil, dass die benötigte Rechenzeit durch die verlängerten Abfragezeiträume stark reduziert wird. Die Software- und Betriebssystementwicklung bietet eine Technik, um die Verwendung von teurem \lstinline{polling} zu umgehen: die Verwendung von \lstinline{interrupts} oder \lstinline{callbacks} (zu deutsch "Rückruf"). Hierbei erhält der Absender der Anfrage eine Rückmeldung, sobald die Antwort vorliegt. Der Vorteil dieser Technik ist, dass nach dem Absenden der Anfrage keine CPU-Zeit seitens des Absenders notwendig ist. Es muss jedoch eine Möglichkeit vorhanden sein, den Absender im Falle einer eingehenden Antwort zum Vorgang zurückzuführen, sodass die weitere Bearbeitung möglich wird. Im Falle der Anbindung der Telegram API übernimmt dies der \lstinline{webhook}. Der \lstinline{webhook} muss (durch die Telegram-API) öffentlich aus dem Internet erreichbar und durch den Verwalter des Bots im Vorhinein durch eine entsprechenden URL definiert worden sein. Wird eine Nachricht an den Bot gesendet, prüft die Telegram-API, intern, ob Definitionen für webhooks vorliegen. Im positiven Fall sendet die Telegram API eine HTTP-Anfrage an die URL des webhooks mit den Details zur eingegangenen Nachricht. Nach Erhalt der Anfrage durch den webhook muss dieser die Daten zur verarbeitenden Software zurückführen und die weitere Bearbeitung auslösen.